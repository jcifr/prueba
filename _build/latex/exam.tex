%% Generated by Sphinx.
\def\sphinxdocclass{report}
\documentclass[letterpaper,10pt,english]{sphinxmanual}
\ifdefined\pdfpxdimen
   \let\sphinxpxdimen\pdfpxdimen\else\newdimen\sphinxpxdimen
\fi \sphinxpxdimen=.75bp\relax

\usepackage[utf8]{inputenc}
\ifdefined\DeclareUnicodeCharacter
 \ifdefined\DeclareUnicodeCharacterAsOptional\else
  \DeclareUnicodeCharacter{00A0}{\nobreakspace}
\fi\fi
\usepackage{cmap}
\usepackage[T1]{fontenc}
\usepackage{amsmath,amssymb,amstext}
\usepackage{babel}
\usepackage{times}
\usepackage[Bjarne]{fncychap}
\usepackage{longtable}
\usepackage{sphinx}

\usepackage{geometry}
\usepackage{multirow}
\usepackage{eqparbox}

% Include hyperref last.
\usepackage{hyperref}
% Fix anchor placement for figures with captions.
\usepackage{hypcap}% it must be loaded after hyperref.
% Set up styles of URL: it should be placed after hyperref.
\urlstyle{same}
\addto\captionsenglish{\renewcommand{\contentsname}{Contents:}}

\addto\captionsenglish{\renewcommand{\figurename}{Fig.}}
\addto\captionsenglish{\renewcommand{\tablename}{Table}}
\addto\captionsenglish{\renewcommand{\literalblockname}{Listing}}

\addto\extrasenglish{\def\pageautorefname{page}}

\setcounter{tocdepth}{2}



\title{exam Documentation}
\date{Oct 10, 2018}
\release{}
\author{mi}
\newcommand{\sphinxlogo}{}
\renewcommand{\releasename}{Release}
\makeindex

\begin{document}

\maketitle
\sphinxtableofcontents
\phantomsection\label{\detokenize{index::doc}}
\begin{sphinxShadowBox}
\sphinxstyletopictitle{Table of Contents}
\begin{itemize}
\item {} 
\phantomsection\label{\detokenize{index:id1}}{\hyperref[\detokenize{index:welcome-to-prueba-s-documentation}]{\sphinxcrossref{Welcome to prueba´s documentation!}}}

\item {} 
\phantomsection\label{\detokenize{index:id2}}{\hyperref[\detokenize{index:index-and-tables}]{\sphinxcrossref{Index and tables}}}

\end{itemize}
\end{sphinxShadowBox}




\chapter{Introduction}
\label{\detokenize{usage/introduction:welcome-to-prueba-s-documentation}}\label{\detokenize{usage/introduction:introduction}}\label{\detokenize{usage/introduction::doc}}
This section is a brief introduction to reStructuredText (reST) concepts and
syntax, intended to provide authors with enough information to author documents
productively.  Since reST was designed to be a simple, unobtrusive markup
language, this will not take too long.


\section{Paragraphs}
\label{\detokenize{usage/introduction:paragraphs}}
The paragraph is the most basic block in a reST document.
Paragraphs are simply chunks of text separated by one or more blank lines.
As in Python, indentation is significant in reST,
so all lines of the same paragraph must be left-aligned to the same level of indentation.


\section{Inline markup}
\label{\detokenize{usage/introduction:inline-markup}}
The standard reST inline markup is quite simple: use
\begin{itemize}
\item {} 
one asterisk: \sphinxstyleemphasis{text} for emphasis (italics),

\item {} 
two asterisks: \sphinxstylestrong{text} for strong emphasis (boldface), and

\item {} 
backquotes: \sphinxcode{text} for code samples.

\end{itemize}

If asterisks or backquotes appear in running text and could be confused with inline markup delimiters,
they have to be escaped with a backslash.

Be aware of some restrictions of this markup:
\begin{itemize}
\item {} 
it may not be nested,

\item {} 
content may not start or end with whitespace: * text* is wrong,

\item {} 
it must be separated from surrounding text by non-word characters.
Use a backslash escaped space to work around that: thisis\sphinxstyleemphasis{one}word.

\end{itemize}

These restrictions may be lifted in future versions of the docutils.

reST also allows for custom “interpreted text roles”,
which signify that the enclosed text should be interpreted in a specific way.
Sphinx uses this to provide semantic markup and cross-referencing of identifiers,
as described in the appropriate section.
The general syntax is \sphinxcode{:rolename:{}`content}.{}`

Standard reST provides the following roles:
\begin{itemize}
\item {} 
emphasis \textendash{} alternate spelling for \sphinxstyleemphasis{emphasis}

\item {} 
strong \textendash{} alternate spelling for \sphinxstylestrong{strong}

\item {} 
literal \textendash{} alternate spelling for \sphinxcode{literal}

\item {} 
subscript \textendash{} subscript text

\item {} 
superscript \textendash{} superscript text

\item {} 
title-reference \textendash{} for titles of books, periodicals, and other materials

\end{itemize}


\section{Lists and Quote-like blocks}
\label{\detokenize{usage/introduction:lists-and-quote-like-blocks}}
List markup is natural:
just place an asterisk at the start of a paragraph and indent properly.
The same goes for numbered lists; they can also be autonumbered using a \# sign:
\begin{itemize}
\item {} 
This is a bulleted list.

\item {} 
It has two items, the second
item uses two lines.

\end{itemize}
\begin{enumerate}
\item {} 
This is a numbered list.

\item {} 
It has two items too.

\item {} 
This is a numbered list.

\item {} 
It has two items too.

\end{enumerate}

Nested lists are possible,
but be aware that they must be separated from the parent list items by blank lines:
\begin{itemize}
\item {} 
this is

\item {} 
a list
\begin{itemize}
\item {} 
with a nested list

\item {} 
and some subitems

\end{itemize}

\item {} 
and here the parent list continues

\end{itemize}

Definition lists are created as follows:
\begin{description}
\item[{term (up to a line of text)}] \leavevmode
Definition of the term, which must be indented

and can even consist of multiple paragraphs

\item[{next term}] \leavevmode
Description.

\end{description}

Note that the term cannot have more than one line of text.

Quoted paragraphs are created by just indenting them more than the surrounding paragraphs.

Line blocks are a way of preserving line breaks:

\begin{DUlineblock}{0em}
\item[] These lines are
\item[] broken exactly like in
\item[] the source file.
\end{DUlineblock}

There are also several more special blocks available:
\begin{itemize}
\item {} 
field lists

\item {} 
option lists

\item {} 
quoted literal blocks

\item {} 
doctest blocks

\end{itemize}


\section{Source Code}
\label{\detokenize{usage/introduction:source-code}}
Literal code blocks (ref) are introduced by ending a paragraph with the special marker ::.
The literal block must be indented (and, like all paragraphs, separated from the surrounding ones by blank lines):

This is a normal text paragraph. The next paragraph is a code sample:

\begin{sphinxVerbatim}[commandchars=\\\{\}]
\PYG{n}{It} \PYG{o+ow}{is} \PYG{o+ow}{not} \PYG{n}{processed} \PYG{o+ow}{in} \PYG{n+nb}{any} \PYG{n}{way}\PYG{p}{,} \PYG{k}{except}
\PYG{n}{that} \PYG{n}{the} \PYG{n}{indentation} \PYG{o+ow}{is} \PYG{n}{removed}\PYG{o}{.}

\PYG{n}{It} \PYG{n}{can} \PYG{n}{span} \PYG{n}{multiple} \PYG{n}{lines}\PYG{o}{.}
\end{sphinxVerbatim}

This is a normal text paragraph again.

The handling of the :: marker is smart:
\begin{itemize}
\item {} 
If it occurs as a paragraph of its own, that paragraph is completely left out of the document.

\item {} 
If it is preceded by whitespace, the marker is removed.

\item {} 
If it is preceded by non-whitespace, the marker is replaced by a single colon.

\end{itemize}

That way, the second sentence in the above example’s first paragraph would be rendered as “The next paragraph is a code sample:”.


\section{Tables}
\label{\detokenize{usage/introduction:tables}}
Two forms of tables are supported.
For grid tables , you have to “paint” the cell grid yourself. They look like this:

\noindent\begin{tabulary}{\linewidth}{|L|L|L|L|}
\hline
\sphinxstylethead{\relax 
Header row, column 1
(header rows optional)
\unskip}\relax &\sphinxstylethead{\relax 
Header 2
\unskip}\relax &\sphinxstylethead{\relax 
Header 3
\unskip}\relax &\sphinxstylethead{\relax 
Header 4
\unskip}\relax \\
\hline
body row 1, column 1
&
column 2
&
column 3
&
column 4
\\
\hline
body row 2
&
...
&
...
&\\
\hline\end{tabulary}


Simple tables (ref) are easier to write, but limited: they must contain more than one row, and the first column cannot contain multiple lines.
They look like this:

\noindent\begin{tabulary}{\linewidth}{|L|L|L|}
\hline
\sphinxstylethead{\relax 
A
\unskip}\relax &\sphinxstylethead{\relax 
B
\unskip}\relax &\sphinxstylethead{\relax 
A and B
\unskip}\relax \\
\hline
False
&
False
&
False
\\
\hline
True
&
False
&
False
\\
\hline
False
&
True
&
False
\\
\hline
True
&
True
&
True
\\
\hline\end{tabulary}



\section{Hyperlinks}
\label{\detokenize{usage/introduction:hyperlinks}}

\subsection{External links}
\label{\detokenize{usage/introduction:external-links}}
Use \sphinxhref{http://example.com/}{Link text} for inline web links.
If the link text should be the web address, you don’t need special markup at all, the parser finds links and mail addresses in ordinary text.

You can also separate the link and the target definition, like this:

This is a paragraph that contains \sphinxhref{http://example.com/}{a link}.


\subsection{Internal links}
\label{\detokenize{usage/introduction:a-link}}\label{\detokenize{usage/introduction:internal-links}}
Internal linking is done via a special reST role provided by Sphinx, see the section on specific markup, Cross-referencing arbitrary locations.


\section{Sections}
\label{\detokenize{usage/introduction:sections}}
Section headers are created by underlining (and optionally overlining) the section title with a punctuation character,
at least as long as the text:

Normally, there are no heading levels assigned to certain characters as the structure is determined from the succession of headings.
However, this convention is used in Python’s Style Guide for documenting which you may follow:
\begin{itemize}
\item {} 
\# with overline, for parts

\item {} \begin{itemize}
\item {} 
with overline, for chapters

\end{itemize}

\item {} 
=, for sections

\item {} 
-, for subsections

\item {} 
\textasciicircum{}, for subsubsections

\item {} 
\sphinxquotedblright{}, for paragraphs

\end{itemize}

Of course, you are free to use your own marker characters (see the reST documentation), and use a deeper nesting level,
but keep in mind that most target formats (HTML, LaTeX) have a limited supported nesting depth.


\section{Citations}
\label{\detokenize{usage/introduction:citations}}
Standard reST citations are supported, with the additional feature that they are “global”, i.e. all citations can be referenced from all files. Use them like so:

Lorem ipsum \phantomsection\label{\detokenize{usage/introduction:id1}}{\hyperref[\detokenize{usage/introduction:ref}]{\sphinxcrossref{{[}Ref{]}}}} dolor sit amet.

Citation usage is similar to footnote usage, but with a label that is not numeric or begins with \#


\section{Comments}
\label{\detokenize{usage/introduction:comments}}
Every explicit markup block which isn’t a valid markup construct (like the footnotes above) is regarded as a comment. For example:

You can indent text after a comment start to form multiline comments:


\section{Source encoding}
\label{\detokenize{usage/introduction:source-encoding}}
Since the easiest way to include special characters like em dashes or copyright signs in reST is to directly write them as Unicode characters,
one has to specify an encoding. Sphinx assumes source files to be encoded in UTF-8 by default;
you can change this with the source\_encoding config value.


\chapter{Section to cross-reference}
\label{\detokenize{usage/introduction:section-to-cross-reference}}
This is the text of the section.

It refers to the section itself, see For more details, see {\hyperref[\detokenize{usage/introduction:introduction}]{\sphinxcrossref{\DUrole{std,std-ref}{Introduction}}}}.

Link to another {\hyperref[\detokenize{usage/introduction:inline-markup}]{\sphinxcrossref{\DUrole{std,std-ref}{Inline markup}}}}.

Link to another {\hyperref[\detokenize{usage/introduction:inline-markup}]{\sphinxcrossref{\DUrole{std,std-ref}{The same}}}} at the same Inline markup.

Link to {\hyperref[\detokenize{usage/images:images}]{\sphinxcrossref{\DUrole{std,std-ref}{Images}}}}.

Link to {\hyperref[\detokenize{usage/images:figures}]{\sphinxcrossref{\DUrole{std,std-ref}{Figures}}}}.


\chapter{Directives}
\label{\detokenize{usage/directives:directives}}\label{\detokenize{usage/directives::doc}}
\begin{sphinxadmonition}{note}{Note:}
This function is not suitable for sending spam e-mails.
\end{sphinxadmonition}

\begin{sphinxadmonition}{warning}{Warning:}
An important bit of information about an API that a user should be very aware of when using whatever bit of API the warning pertains to.
The content of the directive should be written in complete sentences and include all appropriate punctuation.
This differs from note in that it is recommended over note for information regarding security.
\end{sphinxadmonition}

\DUrole{versionmodified}{New in version 2.5: }The \sphinxstyleemphasis{spam} parameter.

\DUrole{versionmodified}{Deprecated since version 3.1: }Use \sphinxcode{spam()} instead.


\sphinxstrong{See also:}

\begin{description}
\item[{Module \sphinxcode{zipfile}}] \leavevmode
Documentation of the \sphinxcode{zipfile} standard module.

\item[{\sphinxhref{http://link}{GNU tar manual, Basic Tar Format}}] \leavevmode
Documentation for tar archive files, including GNU tar extensions.

\end{description}


\paragraph{title}
\begin{itemize}\setlength{\itemsep}{0pt}\setlength{\parskip}{0pt}
\item {} 
A list of

\item {} 
short items

\item {} 
that should be

\item {} 
displayed

\item {} 
horizontally

\end{itemize}
\begin{description}
\item[{environment\index{environment|textbf}}] \leavevmode\phantomsection\label{\detokenize{usage/directives:term-environment}}
A structure where information about all documents under the root is
saved, and used for cross-referencing.  The environment is pickled
after the parsing stage, so that successive runs only need to read
and parse new and changed documents.

\item[{source directory\index{source directory|textbf}}] \leavevmode\phantomsection\label{\detokenize{usage/directives:term-source-directory}}
The directory which, including its subdirectories, contains all
source files for one Sphinx project.

\item[{Sphinx\index{Sphinx|textbf}}] \leavevmode\phantomsection\label{\detokenize{usage/directives:term-sphinx}}
Sphinx is a tool that makes it easy to create intelligent and beautiful documentation.
It was originally created for the Python documentation, and it has excellent facilities for the documentation of software projects in a range of languages.

\item[{RST\index{RST|textbf}}] \leavevmode\phantomsection\label{\detokenize{usage/directives:term-rst}}
RST is an easy-to-read, what-you-see-is-what-you-get plain text markup syntax and parser system.
It is useful for in-line program documentation (such as Python docstrings), for quickly creating simple web pages, and for standalone documents.
RST is designed for extensibility for specific application domains. The RST parser is a component of Docutils.

\item[{Sublime Text\index{Sublime Text|textbf}}] \leavevmode\phantomsection\label{\detokenize{usage/directives:term-sublime-text}}
Sublime Text is a sophisticated text editor for code, markup and prose. You'll love the slick user interface, extraordinary features and amazing performance.

\end{description}

\begin{sphinxadmonition}{note}{Note:}
This function is not suitable for sending spam e-mails.
\end{sphinxadmonition}

How to include images in project:

\noindent{\hspace*{\fill}\sphinxincludegraphics[scale=0.5]{{im1}.jpg}\hspace*{\fill}}


\begin{productionlist}
\phantomsection\label{\detokenize{usage/directives:grammar-token-try_stmt}}\production{try\_stmt}{ try1\_stmt \textbar{} try2\_stmt}
\phantomsection\label{\detokenize{usage/directives:grammar-token-try1_stmt}}\production{try1\_stmt}{ \sphinxquotedblleft{}try\sphinxquotedblright{} \sphinxquotedblright{}:\sphinxquotedblright{} \sphinxcode{suite}}
\productioncont{ (\sphinxquotedblleft{}except\sphinxquotedblright{} {[}\sphinxcode{expression} {[}\sphinxquotedblright{},\sphinxquotedblright{} \sphinxcode{target}{]}{]} \sphinxquotedblright{}:\sphinxquotedblright{} \sphinxcode{suite})+}
\productioncont{ {[}\sphinxquotedblright{}else\sphinxquotedblright{} \sphinxquotedblright{}:\sphinxquotedblright{} \sphinxcode{suite}{]}}
\productioncont{ {[}\sphinxquotedblright{}finally\sphinxquotedblright{} \sphinxquotedblright{}:\sphinxquotedblright{} \sphinxcode{suite}{]}}
\phantomsection\label{\detokenize{usage/directives:grammar-token-try2_stmt}}\production{try2\_stmt}{ \sphinxquotedblleft{}try\sphinxquotedblright{} \sphinxquotedblright{}:\sphinxquotedblright{} \sphinxcode{suite}}
\productioncont{ \sphinxquotedblleft{}finally\sphinxquotedblright{} \sphinxquotedblright{}:\sphinxquotedblright{} \sphinxcode{suite}}
\end{productionlist}


\begin{sphinxadmonition}{danger}{Danger:}
Beware killer rabbits!
\end{sphinxadmonition}

\begin{sphinxShadowBox}
\sphinxstyletopictitle{Topic Title}

Subsequent indented lines comprise
the body of the topic, and are
interpreted as body elements.
\end{sphinxShadowBox}

\begin{sphinxShadowBox}
\sphinxstylesidebartitle{Sidebar Title}
\sphinxstylesidebarsubtitle{Optional Sidebar Subtitle}

Subsequent indented lines comprise
the body of the sidebar, and are
interpreted as body elements.
\end{sphinxShadowBox}

The `rm' command is very dangerous.  If you are logged
in as root and enter

\begin{sphinxVerbatim}[commandchars=\\\{\}]
\PYG{n}{cd} \PYG{o}{/}
\PYG{n}{rm} \PYG{o}{\PYGZhy{}}\PYG{n}{rf} \PYG{o}{*}
\end{sphinxVerbatim}
\noindent
you will erase the entire contents of your file system.

This paragraph might be rendered in a custom way.

Parsing the above results in the following pseudo-XML:
\begin{description}
\item[{\textless{}container classes=\sphinxquotedblright{}custom\sphinxquotedblright{}\textgreater{}}] \leavevmode\begin{description}
\item[{\textless{}paragraph\textgreater{}}] \leavevmode
This paragraph might be rendered in a custom way.

\end{description}

\end{description}

The \sphinxquotedblleft{}container\sphinxquotedblright{} directive is the equivalent of HTML's \textless{}div\textgreater{} element.
It may be used to group a sequence of elements for user- or application-specific purposes.


\chapter{Installation!}
\label{\detokenize{usage/installation:installation}}\label{\detokenize{usage/installation::doc}}\begin{itemize}
\item {} 
This is a bulleted list.

\item {} 
It has two items, the second
item uses two lines.

\end{itemize}
\begin{enumerate}
\item {} 
This is a numbered list.

\item {} 
It has two items too.

\item {} 
This is a numbered list.

\item {} 
It has two items too.

\end{enumerate}
\begin{description}
\item[{term (up to a line of text)}] \leavevmode
Definition of the term, which must be indented

and can even consist of multiple paragraphs

\item[{next term}] \leavevmode
Description.

\end{description}

\begin{DUlineblock}{0em}
\item[] These lines are
\item[] broken exactly like in
\item[] the source file.
\end{DUlineblock}


\chapter{Quickstart}
\label{\detokenize{usage/quickstart::doc}}\label{\detokenize{usage/quickstart:quickstart}}
The standard reST inline markup is quite simple: use
one asterisk: \sphinxstyleemphasis{text} for emphasis (italics),
two asterisks: \sphinxstylestrong{text} for strong emphasis (boldface), and
backquotes: \sphinxcode{text} for code samples.

If asterisks or backquotes appear in running text and could be confused with inline markup delimiters, they have to be escaped with a backslash.

This is a normal text paragraph. The next paragraph is a code sample:

\begin{sphinxVerbatim}[commandchars=\\\{\}]
\PYG{n}{It} \PYG{o+ow}{is} \PYG{o+ow}{not} \PYG{n}{processed} \PYG{o+ow}{in} \PYG{n+nb}{any} \PYG{n}{way}\PYG{p}{,} \PYG{k}{except}
\PYG{n}{that} \PYG{n}{the} \PYG{n}{indentation} \PYG{o+ow}{is} \PYG{n}{removed}\PYG{o}{.}

\PYG{n}{It} \PYG{n}{can} \PYG{n}{span} \PYG{n}{multiple} \PYG{n}{lines}\PYG{o}{.}
\end{sphinxVerbatim}

This is a normal text paragraph again.

\noindent\begin{tabulary}{\linewidth}{|L|L|L|L|}
\hline
\sphinxstylethead{\relax 
Header row, column 1
(header rows optional)
\unskip}\relax &\sphinxstylethead{\relax 
Header 2
\unskip}\relax &\sphinxstylethead{\relax 
Header 3
\unskip}\relax &\sphinxstylethead{\relax 
Header 4
\unskip}\relax \\
\hline
body row 1, column 1
&
column 2
&
column 3
&
column 4
\\
\hline
body row 2
&
...
&
...
&\\
\hline\end{tabulary}



\chapter{This is a tables}
\label{\detokenize{usage/tablas::doc}}\label{\detokenize{usage/tablas:this-is-a-tables}}

\section{Tables}
\label{\detokenize{usage/tablas:tables}}

\subsection{Tables 1}
\label{\detokenize{usage/tablas:tables-1}}

\subsubsection{H4 -- Subsubsection}
\label{\detokenize{usage/tablas:h4-subsubsection}}

\begin{threeparttable}
\capstart\caption{Truth table for \sphinxquotedblleft{}not\sphinxquotedblright{}}\label{\detokenize{usage/tablas:id1}}
\noindent\begin{tabulary}{\linewidth}{|L|L|}
\hline
\sphinxstylethead{\relax 
A
\unskip}\relax &\sphinxstylethead{\relax 
not A
\unskip}\relax \\
\hline
False
&
True
\\
\hline
True
&
False
\\
\hline\end{tabulary}

\end{threeparttable}


\noindent\begin{tabulary}{\linewidth}{|L|L|L|L|}
\hline
\sphinxstylethead{\relax 
Header row, column 1
(header rows optional)
\unskip}\relax &\sphinxstylethead{\relax 
Header 2
\unskip}\relax &\sphinxstylethead{\relax 
Header 3
\unskip}\relax &\sphinxstylethead{\relax 
Header 4
\unskip}\relax \\
\hline
body row 1, column 1
&
column 2
&
column 3
&
column 4
\\
\hline
body row 2
&
...
&
...
&\\
\hline\end{tabulary}


The tables are as seen

\noindent\begin{tabulary}{\linewidth}{|L|L|L|}
\hline
\sphinxstylethead{\relax 
A
\unskip}\relax &\sphinxstylethead{\relax 
B
\unskip}\relax &\sphinxstylethead{\relax 
A and B
\unskip}\relax \\
\hline
False
&
False
&
False
\\
\hline
True
&
False
&
False
\\
\hline
False
&
True
&
False
\\
\hline
True
&
True
&
True
\\
\hline\end{tabulary}


Use \sphinxhref{https://domain.invalid/}{Link text} for inline web links. If the link text should be the web address, you don’t need special markup at all, the parser finds links and mail addresses in ordinary text.

This is a paragraph that contains \sphinxhref{https://domain.invalid/}{a link}.

The handling of the :: marker is smart:

If it occurs as a paragraph of its own, that paragraph is completely left out of the document.
If it is preceded by whitespace, the marker is removed.
If it is preceded by non-whitespace, the marker is replaced by a single colon.
That way, the second sentence in the above example’s first paragraph would be rendered as “The next paragraph is a code sample:”.


\section{Doctest blocks}
\label{\detokenize{usage/tablas:doctest-blocks}}
\begin{sphinxVerbatim}[commandchars=\\\{\}]
\PYG{g+gp}{\PYGZgt{}\PYGZgt{}\PYGZgt{} }\PYG{l+m+mi}{1} \PYG{o}{+} \PYG{l+m+mi}{1}
\PYG{g+go}{2}
\end{sphinxVerbatim}


\section{Definition lists}
\label{\detokenize{usage/tablas:definition-lists}}\begin{description}
\item[{term (up to a line of text)}] \leavevmode
Definition of the term, which must be indented

and can even consist of multiple paragraphs

\item[{next term}] \leavevmode
Description.

\end{description}


\section{Broken lines}
\label{\detokenize{usage/tablas:broken-lines}}
\begin{DUlineblock}{0em}
\item[] These lines are
\item[] broken exactly like in
\item[] the source file.
\end{DUlineblock}


\chapter{Images}
\label{\detokenize{usage/images:images}}\label{\detokenize{usage/images::doc}}
How to include images in project:

\noindent{\hspace*{\fill}\sphinxincludegraphics[scale=0.5]{{im1}.jpg}\hspace*{\fill}}


\section{Figures}
\label{\detokenize{usage/images:figures}}\begin{figure}[htbp]
\centering
\capstart

\noindent\sphinxincludegraphics[scale=0.3]{{im1}.jpg}
\caption{This is the caption of the figure (a simple paragraph).}
\begin{sphinxlegend}
The legend consists of all elements after the caption.  In this
case, the legend consists of this paragraph and the following
table:

\noindent\begin{tabulary}{\linewidth}{|L|L|}
\hline
\sphinxstylethead{\relax 
Symbol
\unskip}\relax &\sphinxstylethead{\relax 
Meaning
\unskip}\relax \\
\hline
\noindent\sphinxincludegraphics{{im1}.jpg}
&
Campground
\\
\hline
\noindent\sphinxincludegraphics{{im1}.jpg}
&
Lake
\\
\hline
\noindent\sphinxincludegraphics[scale=0.1]{{im1}.jpg}
&
Mountain
\\
\hline\end{tabulary}

\end{sphinxlegend}
\label{\detokenize{usage/images:id1}}\end{figure}


\chapter{Index and tables}
\label{\detokenize{index:index-and-tables}}\begin{itemize}
\item {} 
\DUrole{xref,std,std-ref}{genindex}

\item {} 
\DUrole{xref,std,std-ref}{modindex}

\item {} 
\DUrole{xref,std,std-ref}{search}

\end{itemize}

\begin{sphinxthebibliography}{Ref}
\bibitem[Ref]{\detokenize{Ref}}{\phantomsection\label{\detokenize{usage/introduction:ref}} 
Book or article reference, URL or whatever.
}
\end{sphinxthebibliography}



\renewcommand{\indexname}{Index}
\printindex
\end{document}